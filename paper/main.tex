\documentclass[conference]{IEEEtran}
\IEEEoverridecommandlockouts
% The preceding line is only needed to identify funding in the first footnote. If that is unneeded, please comment it out.
\usepackage{cite}
\usepackage{amsmath,amssymb,amsfonts}
\usepackage{algorithmic}
\usepackage{graphicx}
\usepackage{textcomp}
\usepackage{xcolor}
\def\BibTeX{{\rm B\kern-.05em{\sc i\kern-.025em b}\kern-.08em
    T\kern-.1667em\lower.7ex\hbox{E}\kern-.125emX}}
\begin{document}

\title{Robust Audio Deepfake Detection}

\author{\IEEEauthorblockN{Thomas Barand}
\IEEEauthorblockA{\textit{Ecole Centrale de Lyon} \\
Lyon, France \\
}
\and
\IEEEauthorblockN{Aurelien Laouar}
\IEEEauthorblockA{\textit{Ecole Centrale de Lyon} \\
Lyon, France \\
}
\and
\IEEEauthorblockN{Emile Dugelay}
\IEEEauthorblockA{\textit{Ecole Centrale de Lyon} \\
Lyon, France \\
}
\and
\IEEEauthorblockN{Baptiste Campeas}
\IEEEauthorblockA{\textit{Ecole Centrale de Lyon} \\
Lyon, France \\
}
}

\maketitle

\begin{abstract}
Abstract
\end{abstract}

\begin{IEEEkeywords}
signal processing, genAI
\end{IEEEkeywords}

\section{Introduction}
Introduction.

\begin{thebibliography}{00}
\bibitem{b1} A. Name, B. Name, and C. Name, ``Exemple Reference'' Source, vol. A247, pp. 529--551, April 1955.
\end{thebibliography}

\end{document}